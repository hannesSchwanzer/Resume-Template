\documentclass{resume}
\usepackage[utf8]{inputenc}
\usepackage[german]{babel}

\usepackage{blindtext}

\name{Hannes Schwanzer}
\address{00000}{Test}{Teststrasse}{8}
\contact{+49 000 000000}{test.test@test.com}
\link{Github}{https://github.com/hannesSchwanzer}

\begin{document}
\maketitle

% Ausbildung
\begin{cvsection}{Ausbildung}
  \begin{cvsubsection}{Fachoberschule Ansbach}{2017 -- 2019}{Allgemeine Hochschulreife}
  \end{cvsubsection}
  \begin{cvsubsection}{Hochschule RheinMain}{Oktober 2020 -- Juli 2024}{B.Sc. Angewandte Informatik}
	\item Gesamtnote: 1,2
  \end{cvsubsection}
\end{cvsection}


% Berufliche Erfahrung
\begin{cvsection}{Berufliche Erfahrung}
  \begin{cvsubsection}{Tutor}{Oktober 2021 -- Februar 2023}{Hochschule RheinMain}
      \item Tutorienleiter in Objektorientierte Softwarentwicklung und Programmiermethoden \& Techniken
      \item Praktikumsbegleiter in Datenbanken und Objektorientierte Softwarentwicklung
  \end{cvsubsection}
  \begin{cvsubsection}{Pflichtpraktikum}{April 2023 -- August 2023}{AOE GmbH}
      \item Backendentwicklung in Kotlin und Spring Boot
      \item Erstellung von Docker-Images f\"ur Microservices
      \item Arbeiten mit MongoDB und MySQL
  \end{cvsubsection}
\end{cvsection}


% Hochschulprojekte
\begin{cvsection}{Hochschulprojekte}
  \begin{cvproject}
  {Blutspendeapp}{C\#, ASP.NET Core, Entity Framework}{Oktober 2022 - Februar 2023}
	\item App zur Benachrichtigung über den Bedarf einer Blutspende
	\item Entwicklung der Backend- und Datenbankanbindung
  \end{cvproject}
  \begin{cvproject}
  {Question Answering System}{Python, Pytorch, Elasticsearch}{Oktober 2022 - Februar 2023}
	\item Entwicklung eines vollst\"andigen QA-Systems
	\item Antwort auf simple Fragen aus WikiBase
  \end{cvproject}
  \begin{cvproject}
  {Bachelorthesis}{Python, Sklearn, NumPy, Matplotlib}{Oktober 2022 - Februar 2023}
	\item Untersuchung verschiedener expliziten Relevance Feedback Verfahren f\"ur kontextualisierte Code Suche
	\item Einbindung in bestehendes FastAPI Backend und Vue.js Frontend
  \end{cvproject}
\end{cvsection}


\begin{cvsection}{Skills}
  \begin{cvskills}
    \cvskill{Programmiersprachen}{Python, Java, Kotlin, C\#, C, Javascript}
    \cvskill{Frameworks}{Spring Boot, JUnit, ASP.NET Core}
    \cvskill{Libraries}{Sklearn, NumPy, Matplotlib}
    \cvskill{Developer Tools}{Git, Docker, Gitlab CI\/CD,VS Code, Visual Studio,IntelliJ, Neovim}
  \end{cvskills}
\end{cvsection}


\begin{cvsection}{Sprachkenntniss}
  \begin{cvskills}
    \cvskill{Deutsch}{Muttersprache}
    \cvskill{Englisch}{Konversationssicher}
  \end{cvskills}
\end{cvsection}

\end{document}
